\documentclass[acmsmall]{acmart}

\usepackage{booktabs}
% \usepackage{subcaption}
\usepackage{amsthm,mathtools,amssymb,stmaryrd}
\usepackage{amsfonts}
\usepackage{graphicx}
\usepackage{listings}
\usepackage{url}
\usepackage{hyperref}
\usepackage{xspace}
\usepackage{multirow}
\usepackage{textcomp}
\usepackage[T1]{fontenc}
\usepackage[scaled=0.75]{beramono}
\usepackage{paralist}
\usepackage[inference]{semantic}
% \usepackage[ruled]{algorithm2e} % For algorithms
\usepackage{algorithmicx}
\usepackage[noend]{algpseudocode}
\usepackage{wrapfig}
% \usepackage{auto-pst-pdf} % attempt to include ps file
\usepackage{pdfpages}

\usepackage[version=0.96]{pgf}
\usepackage{tikz}
\usetikzlibrary{arrows,shapes,automata,backgrounds,petri,positioning}
\usepackage{todonotes}

\usepackage{ltablex,array}

\begin{document}


% Text abbreviations
\newcommand{\etc}{\emph{etc}\xspace}
\newcommand{\wrt}{\emph{wrt.}\xspace}
\newcommand{\ie}{\emph{i.e.}\xspace}
\newcommand{\Ie}{\emph{I.e.}\xspace}
\newcommand{\eg}{\emph{e.g.}\xspace}
\newcommand{\Eg}{\emph{E.g.}\xspace}
\newcommand{\etal}{\emph{et~al.}\xspace}
\newcommand{\tygar}{\tname{TYGAR}}
\newcommand{\emphbf}[1]{\emph{\textbf{#1}\xspace}}
\newcommand{\mypara}[1]{\smallskip\noindent\emphbf{#1.}\xspace}
\newcommand{\setof}[1]{\ensuremath{\{ #1 \}}}
\newcommand{\sem}[1]{\ensuremath{\llbracket #1 \rrbracket}}

\newcommand{\mFixType}{\emph{Baseline}\xspace}
\newcommand{\mNoGar}{\emph{NOGAR}\xspace}
\newcommand{\mTopType}{\emph{\tygar-0}\xspace}
\newcommand{\mQryType}{\emph{\tygar-Q}\xspace}
\newcommand{\mQryTypeBounded}{\emph{\tygar-QB}\xspace}

% Remarks
\newif\ifdraft
\draftfalse

\ifdraft
\newcommand\authorrnote[3]{\textcolor{#1}{({#2}: {#3})}\xspace}
\newcommand{\TODO}[1]{{\color{orange!80!black}[\textsl{#1}]}}
\else
\newcommand\authorrnote[2]{}
\newcommand{\TODO}[1]{}
\fi

% Tool names
\newcommand{\nex}{\ensuremath{E}}
\newcommand{\tname}[1]{\textsc{#1}\xspace}
\newcommand{\tool}{\tname{H+}}
\newcommand{\sys}{\tool}
\newcommand{\hoogle}{\tname{Hoogle}}
\newcommand{\agda}{\tname{Agda}}
\newcommand{\sypet}{\tname{SyPet}}
\newcommand{\insynth}{\tname{InSynth}}
\newcommand{\prospector}{\tname{Prospector}}
\newcommand{\blaze}{\tname{Blaze}}
\newcommand{\djinn}{\tname{Djinn}}
\newcommand{\synquid}{\tname{Synquid}}
\newcommand{\myth}{\tname{Myth2}}
\newcommand{\corelang}{$\lambda_H$\xspace}

\definecolor{commentgreen}{rgb}{0.25,0.5,0.35}

% Listings
\lstset{
  language=Haskell,
  basicstyle=\small\ttfamily,
  numbers=none,
  escapeinside={(*@}{@*)},
  keywordstyle=[1]{\ttfamily},
  commentstyle={\color{commentgreen}},
  literate=%
    {->}{$\rightarrow$}2
    {forall}{$\forall$}1
    {_a}{$\alpha$}1
    {_b}{$\beta$}1
    {_t}{$\tau$}1
    {_n}{$\nu$}1
}

\lstdefinestyle{numbers}
{
  numbers=left,
  numberstyle=\sf,
  xleftmargin=15pt
}

\newcommand{\T}[1]{\mbox{\lstinline[language=Haskell,basicstyle=\ttfamily,columns=fixed]^#1^}}

% TikZ

\tikzstyle{place}=[circle,thick,draw=blue!75,fill=blue!20,minimum size=6mm]
\tikzstyle{final}=[place,double]
\tikzstyle{other}=[place,draw=red!75,fill=red!20]
\tikzstyle{blank}=[minimum size=6mm]
\tikzstyle{transition}=[rectangle,thick,draw=black!75,fill=black!20,minimum size=4mm]
\tikzstyle{sol}=[very thick]
\tikzstyle{mismatch}=[thick,red!75]

\newcommand{\labels}[2]{{\begin{tabular}{c} \textcolor{blue}{#1} \\ \textcolor{red}{#2} \end{tabular}}}
\newcommand{\labelc}[1]{{\textcolor{red}{#1}}}

% Algorithms

\newcommand*\Let[2]{\State #1 $\gets$ #2}
\algrenewcommand\algorithmicrequire{\textbf{Input:}}
\algrenewcommand\algorithmicensure{\textbf{Output:}}
\algnewcommand{\LineFor}[2]{
  \State\algorithmicfor\ {#1}\ \algorithmicdo\ {#2}}
% \algnewcommand\algorithmicswitch{\textbf{match}}
% \algnewcommand\algorithmiccase{\textbf{case}}
% \algdef{SE}[SWITCH]{Switch}{EndSwitch}[1]{\algorithmicswitch\ #1}{\algorithmicend\ \algorithmicswitch}%
% \algdef{SE}[CASE]{Case}{EndCase}[1]{\algorithmiccase\ #1:}{\algorithmicend\ \algorithmiccase}%
% \algtext*{EndSwitch}%
% \algtext*{EndCase}%

\newcommand{\abssynth}{SynAbstract}
\newcommand{\shortestpath}{ShortestValidPath}

% Math

\newcommand{\dom}[1]{\mathsf{dom}(#1)}
\newcommand{\rng}[1]{\mathsf{range}(#1)}
\newcommand{\subst}[2]{{#1} \mapsto {#2}}
\newcommand{\pre}{\mathsf{pre}}
\newcommand{\post}{\mathsf{post}}
\newcommand{\parents}{\mathsf{parents}}
\newcommand{\many}[1]{\overline{#1}}
\newcommand{\tvars}[1]{\mathsf{tv}(#1)}
\newcommand{\ty}[1]{\Lambda(#1)}
\newcommand{\abset}{\mathcal{A}}
\newcommand{\wrap}[2]{\mathsf{wrap}(#1,#2)}
\newcommand{\nat}{\mathbb{N}}
\newcommand{\net}{\mathcal{N}}
\newcommand{\steps}[1]{\xrightarrow{#1}}
\newcommand{\toBase}[1]{\mathsf{base}(#1)}
\newcommand{\fo}{\mathsf{desugar}}
\newcommand{\hask}[1]{\tilde{#1}}
\newcommand{\fromPath}[1]{\mathsf{terms}(#1)}
\newcommand{\fresh}[1]{\mathsf{fresh}(#1)}
\newcommand{\subterms}[1]{\mathsf{subterms}(#1)}
\newcommand{\close}[1]{\mathsf{close}(#1)}
\newcommand{\up}[1]{\mathsf{up}(#1)}
\newcommand{\init}{\mathsf{init}}
\newcommand{\final}{\mathsf{fin}}
\newcommand{\refines}{\preceq}
\newcommand{\inv}{\mathcal{I}}
\newcommand{\sub}{\sqsubseteq}
\newcommand{\meet}{\sqcap}
\newcommand{\join}{\sqcup}
\newcommand{\mgu}[1]{\mathsf{mgu}(#1)}
\newcommand{\bases}{\ensuremath{\mathbf{B}}}
\newcommand{\basebots}{\ensuremath{\mathbf{B_{\bot}}}}
\newcommand{\anew}{A_{\mathit{new}}}


\newcommand{\eapp}[2]{{#1}\ {#2}}
\newcommand{\elam}[2]{\lambda {#1}.{#2}}
\newcommand{\ebody}{e_{\mathit{body}}}
\newcommand{\sapp}[2]{{#1}{#2}}

\newcommand{\memph}[1]{\textcolor{red}{#1}}
\newcommand{\mnoemph}[1]{\textcolor{black}{#1}}

\newcommand{\chkarrow}{\Longleftarrow}
\newcommand{\synarrow}{\Longrightarrow}

\newcommand{\jtyping}[3]{\ensuremath{\Lambda;#1 \vdash #2 :: #3}}
\newcommand{\jtcheck}[3]{\ensuremath{\Lambda;#1 \vdash #2 \chkarrow #3}}
\newcommand{\jtinfer}[3]{\ensuremath{\Lambda;#1 \vdash #2 \synarrow #3}}
\newcommand{\jtinferlib}[4]{\ensuremath{#1;#2 \vdash #3 \synarrow #4}}
\newcommand{\jntyping}[3]{\ensuremath{\Lambda;#1 \not\vdash #2 \chkarrow #3}}
% \newcommand{\jntyping}[3]{\ensuremath{\Lambda;#1 \vdash #2 \chkarrow \bot }}
\newcommand{\jwf}[1]{\Lambda \vdash #1}
\newcommand{\jwfenv}[1]{\Lambda \vdash #1}
\newcommand{\jacheck}[4]{\ensuremath{\Lambda;#1 \vdash_{#4} #2 \chkarrow #3}}
\newcommand{\jainfer}[4]{\ensuremath{\Lambda;#1 \vdash_{#4} #2 \synarrow #3}}
\newcommand{\jancheck}[4]{\ensuremath{\Lambda;#1 \not\vdash_{#4} #2 \chkarrow #3}}
% \newcommand{\jancheck}[4]{\ensuremath{\Lambda;#1 \vdash_{#4} #2 \chkarrow \bot}}
\newcommand{\jawf}[2]{\Lambda;#1 \vdash #2}
\newcommand{\unifies}[2]{#1 \sim #2}
\newcommand{\nunifies}[2]{#1 \not\sim #2}

%% Depricated
\newcommand{\jatyping}[4]{\Lambda;#1 \vdash_{#4} #2 :: #3}
% \newcommand{\aunifies}[3]{#1 \sim #2 \rightsquigarrow #3}

% Evaluation
\newcommand{\componentCount}{291\xspace}
\newcommand{\benchmarkCount}{44\xspace}
\newcommand{\hoogleOnlyBms}{24\xspace}
\newcommand{\SOOnlyBms}{6\xspace}
\newcommand{\ourBms}{17\xspace}

\newcommand{\timeout}{60\xspace}
\newcommand{\qualitytimeout}{100\xspace}
\newcommand{\failureCount}{1\xspace}

\newcommand{\qualityH}{\T{H+}\xspace}
\newcommand{\qualityHD}{\T{H-D}\xspace}
\newcommand{\qualityHR}{\T{H-R}\xspace}
\newcommand{\qualityHRankAvg}{1.6\xspace}
\newcommand{\qualityHDRankAvg}{1.5\xspace}
\newcommand{\qualityHRRankAvg}{0.6\xspace}

\newcommand{\firstSolutionTime}{1.4\xspace}
\newcommand{\nogarTime}{0.6\xspace}
\newcommand{\tygarQTime}{1.2\xspace}
\newcommand{\tygarZTime}{1.4\xspace}

\newcommand{\tygarQBSolnCount}{43\xspace}
\newcommand{\tygarQSolnCount}{34\xspace}
\newcommand{\tygarZSolnCount}{35\xspace}
\newcommand{\nogarSolnCount}{37\xspace}

\newcommand{\applyNTimesRank}{10\xspace}
\newcommand{\applyNTimesPosition}{11\xspace}
\newcommand{\firstJustRank}{18\xspace}
\newcommand{\firstJustPosition}{19\xspace}
\newcommand{\lookupRank}{1\xspace}
\newcommand{\lookupPosition}{18\xspace}

\newcommand{\interestingCount}{32\xspace}


\begin{figure}
    \resizebox{\textwidth}{!}{
    \input{table_results}
    }

  \caption{\tool synthesis times and solution quality on \benchmarkCount benchmarks.
  Suffixes `-QB10', `-Q', `-0', `-NO'
  correspond to four variants of the search algorithm:
  \mQryTypeBounded$[10]$, \mQryType, \mTopType, and \mNoGar.
  Prefixes `t-', `st-`, and `tc-` denote, respectively,
  the total time to a first solution,
  time spend in the SMT solver,
  and time spent type checking (including demand analysis).
  All times are in seconds.
  Absence indicates no solution found within the timeout of \timeout seconds.
  % `tr-' are the number of transitions at the end of search--either after a
  % timeout of \timeout seconds or when the first solution was reached.
  `H' is the number of interesting solutions in the first five on \tool.
  `H-D' is the same number with the demand analyzer disabled.
  `H-R' is the same number with structural typing over relevant typing.
  Absence indicates no interesting solution was in the first five.
  }
\end{figure}
\end{document}
